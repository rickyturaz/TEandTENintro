\part{Rooms: geometry and design}

\chapter{Introduction}

\section{Blocks, sectors and clicks}

\textbf{Blocks, sequares, sectors and clicks}: get used to these terms because you'll hear them frequently. The Tomb Editor is designed to work with a basic \emph{building block}, proportioned to Lara and her movements.
\par Levels are built by connecting a series of rooms comprised of walls and building blocks. The floor and ceiling of these rooms are sectioned into squares or sectors. The building blocks are created when you raise a square up from the floor or lower one down from the ceiling. Four mouse clicks up or down equals the width of these squares sections and creates a perfect cube \emph{Remember all those "blocks" Lara pushed and pulled around?!}
\par The building blocks are not limited to cubes and columns with flat tops. Corners of the surfaces can be pulled up or down to create angled slopes and \emph{organic} surfaces - great for creating rocky caves or sand dunes. \cite{trle_manual}

\section{Basic definitions}

Let's start with the default room: in the previous part, we've learnt that when we create a new level, TombIDE creates for us a level containing one room having size 18x18x3. Ok but.. 18x18x3 what?

It's time to introduce some formal definitions, forming a common dictionary.

\begin{remark}
    Property of a room is its area, measured in \emph{sectors} or \emph{squares}.
\end{remark}

For our default room, its area is 18 squares. We can see it looking at the Sector Options, or switching to the 2D mode in Tomb Editor.

\begin{remark}
    Given a room, the area of its floor is equal to the area of its ceiling.
\end{remark}

\begin{remark}
    Number of floor sectors is equal to number of ceiling sectors.
\end{remark}

Now, recall in mind a way to define position of objects in the three-dimensional space we live our life: the cartesian coordinate system, where we use the three axes \( x, y, z\).
Up to now, we cared about \(x\) and \(y\). Let's introduce the way we can set \(z\) coordinate (height) and so, clicks:

\begin{remark}
    A \textbf{click} is the minimum linear transformation can be applied to a floor or a ceiling sector to increase or decrease its height.
\end{remark}

By using clicks, we can make a block.

\begin{remark}
    A block is the solid shape we can obtain by applying clicks to floor (raising) or ceiling (lowering).
\end{remark}

Finally, we reach the definition of the \emph{Tomb Raider cube}:
\begin{remark}
    A \emph{Tomb Raider cube} is a cube-shaped block, where every edge is 4 clicks long.
\end{remark}

The definition of the Tomb Raider cube is useful because being the easiest possible 3D solid, it's a simple way to remember Lara's sizes compared to her world and the real world.
Let's compare Lara's sizes and her movements with the cube, to better understand.
% Riscrivi questa frase per chiarirla meglio e rivedi il contenuto di questa sezione per vedere se si capisce...