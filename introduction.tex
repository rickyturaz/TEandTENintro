\part{Introduction}
\chapter{History of Tomb Raider Level Editors}
\section{Back to 2000: \emph{Tomb Raider Level Editor} \index{Tomb Raider Level Editor}\index{TRLE}}
Tomb Raider marked a sensational new approach to 3rd person gaming. Fans not only fell in love with Lara and her moves, but with the imaginative and intriguing worlds of her adventures. It all started with Lara's visit to some Egyptian ruins back in 1996, and now with the release of the Tomb Raider Level Editor has come full circle, offering a different sort of adventure in another Egyptian setting. \emph{Tomb Raider Chronicles} marks the end of the line of Tomb Raider games made with these development tools; but rather than viewing this as an end, the release of the editor makes it seem more like a beginning...
\par The \textbf{Tomb Raider Level Editor (TRLE)} includes a tutorial that will walk you through the basics needed to create your own stand alone Tomb Raider levels (but please read the legal disclaimer about commercial use of this product). Even though you will not be able to edit objects or animations (that means Lara’s outfits), you have a wonderful variety of object sets from which to choose. You can sculpt and design on many different ‘levels’ – trigger events, create awe-inspiring spaces, simple to complex…and as you experiment you will learn more about what can be done, and quite possibly discover new methods of applying the knowledge you have acquired.
\par We sincerely hope you will enjoy inventing, designing, and building with and for Lara as much as we have over the past 4 years. We thank all those who have held the enthusiasm for the Tomb Raider series, thereby contributing to its success. We wish you happy adventuring with Lara and the tools used to create her worlds.
\cite{trle_manual}
\section{Paolone's \emph{Next Generation Level Editor} \index{NGLE} \index{Next Generation Level Editor}}
The \textbf{Next Generation Level Editor}, often abbreviated \textbf{NGLE}, is a modified version of the Tomb Raider Level Editor, created by Paolone, and released in January 2007. \cite{wikiraider_NGLE}
\par Tomb Raider Next Generation \index{Tomb Raider Next Generation} (TRNG) \index{TRNG} tools, improve the TRLE tools used to build custom levels with the engine, supplied by Eidos, of Tomb Raider - The Last Revelation.
\par Many objects have beed added, some imported by other TR adventures, like boat or frog-man, other builded ex-novo like Detector or Elevator.
\par There is a new scripter program named NG\_Center. This program other to build your script.dat supplies other little tools. \cite{paolone_trng}
\section{MontyTRC89's \emph{Tomb Editor}}
\textbf{Tomb Editor (TE)} is a level editor designed for the full range of classic Tomb Raider game series (1-5), as well as for contemporary engine reimplementation projects and game engines designed for community modding and level building. \cite{TE_github}

\chapter{Basic concepts to know about Tomb Raider}
% Inserire qui i concetti fondamentali relativi a Tomb Raider, da sapere. Ad esempio, se parlo di TR4 devi sapere di cosa si parla. Scherzaci un po' su: se sei qui a leggere questo manuale, un po' Lara ti deve piacere, no?! :)

\chapter{Engines}
% Spiega la differenza fra editor ed engine, va spiegata. Spiega i vari engine disponibili.

\chapter{Installing Tomb Editor}
First of all, you need to download and install the Tomb Editor pack on your computer. It is available eg. \href{https://tombengine.com/}{here}.
\par The default route of Tomb Editor installed is \path{C:\Tomb Editor}. The contents of this main folder are:
\begin{itemize}
    \item Tomb Editor program.
    \item Side programs dedicated to Tomb Editor: SoundTool, TombIDE, WadTool.
    \item Most of the files which are necessary to start a basic project and level for Tomb Engine. (But texture files for room faces must be find somewhere else. But this is still not necessary now, when you start reading this tutorial.)
    \item Other important files for Tomb Editor pack.
\end{itemize}
So when you have the Tomb Editor pack installed on your computer, then you are just ready to start building levels for Tomb Engine. \cite{akyv_tutorial}

\chapter{Tools}
% Spiega la differenza fra editor ed engine, va spiegata. Spiega i vari engine disponibili.